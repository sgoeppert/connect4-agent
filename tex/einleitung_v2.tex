\section{Einleitung}
\par 
Brettspiele sind nicht nur schöne, soziale Aktivitäten, sie eignen sich auch sehr gut um neue Algorithmen zu erfinden und erproben. Viele Brettspiele sind in ihrer Einfachheit so überschaubar, dass auch als menschlicher Spieler ein perfektes Spiel möglich ist. Das Spiel Tic-Tac-Toe zum Beispiel kann ohne weiteres perfekt gespielt werden, sodass es immer in einem Unentschieden endet. Komplexere Spiele wie Vier Gewinnt, Schach und Go sind dagegen so viel schwieriger, dass es einem menschlichen Spieler nicht möglich ist, immer die spieltheoretisch besten Spielzüge zu spielen. Für ein solches perfektes Spiel sind spezielle Algorithmen und Computerprogramme von Nöten.
\par 
Dass ein perfektes Spiel möglich ist, haben Victor Allis\autocite{allisKnowledgeBasedApproachConnectFour1988} und James D. Allen\autocite{allenExpertPlayConnectFour} für Vier Gewinnt unabhängig voneinander in 1988 gezeigt. Auch wenn Schach noch nicht gelöst ist, gelang es dem Schachcomputer IBM Deep Blue 1997 den amtierenden Schachweltmeister Garry Kasparov in einem Match über sechs Spiele zu schlagen\autocite{IBM100DeepBlue2012}. Nach Schach galt für lange Zeit das chinesische Brettspiel Go als nächster großer Meilenstein für künstliche Intelligenz.
\par 
Go wird auf einem Raster mit 19x19 Linien gespielt. Die Spieler, Schwarz und Weiß, setzen abwechselnd jeweils einen Stein auf eine der Kreuzungen. Das Ziel ist es, Gruppen zu Bilden, die gegnerische Steine oder freies Gebiet umgrenzen, und am Ende das meiste Gebiet zu besitzen. Durch die Größe des Spielfeldes und die vergleichsweise langen Spiele, war Go für lange Zeit außerhalb der Reichweite traditioneller Algorithmen. Nur auf kleinen Spielfeldern und mit zusätzlichen Steinen als Handicap hatten diese Programme eine Chance.\autocite{burnmeisterCSTR339ComputerGo}
\par 
Die Erfindung der Monte Carlo Baumsuche(MCTS) durch Rémi Coulom in 2006\autocite{coulomEfficientSelectivityBackup2007} und die weiteren Arbeiten von Kocsis und Szepesvári\autocite{kocsisBanditBasedMonteCarlo2006} sowie Gelly und Silver \autocite{gellyCombiningOnlineOffline2007} haben die Entwicklung von Computer Go Programmen revolutioniert. Heute wird MCTS in den meisten Go Programmen eingesetzt, und auch das revolutionäre AlphaGo von Google Deepmind\autocite{silverMasteringGameGo2016} setzt die Monte Carlo Baumsuche mit großem Erfolg ein. So gelang es AlphaGo im Oktober 2015 den Europameister Fan Hui fünf zu null zu besiegen, sowie ein halbes Jahr später im März 2016 den Gewinner von 18 Weltmeistertiteln, Lee Sedol, vier zu eins zu besiegen. \autocite{AlphaGoStoryFar}