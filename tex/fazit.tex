\section{Fazit und Ausblick}
\label{chap:fazit}
\improvement[inline]{Das Fazit muss noch geschrieben werden}
In dieser Arbeit habe ich die Monte-Carlo-Baumsuche in Python implementiert und einige Verbesserungen mit Blick auf ihren Nutzen für Vier gewinnt verglichen.
Allgemein lässt sich sagen, dass die Monte-Carlo-Baumsuche gut für solche Spiele geeignet ist.
Sie ist sehr flexibel und kann ohne große Anpassungen mit variablen Spielgrößen Arbeiten.
Ihre Unabhängigkeit von einer konkreten Bewertungsfunktion ist ein Segen für Spiele mit sich verändernden Parametern.

Vier gewinnt profitiert sehr von Modifikationen, die die Simulationsphase verbessern


\subsection{Ausblick}
AlphaZero hat gezeigt, wie gut ein Spieler dieser Architektur durch reines Selfplay lernen kann.
Meine implementierung des neuronalen Netzes ähnelt eher einem Supervised Learning Problem, AlphaZero hat eine komplette Trainings-Pipeline mit hunderten Spielern, die die trainierten Netzwerke verwenden um laufend neue Trainingsdaten zu erzeugen.
Dies war mit meinen begrenzten Mitteln leider nicht möglich, weshalb ich mich dafür entschied die Trainingsdaten separat zu erzeugen, bevor ich mit dem Training der Netzwerke beginne.

Bei meinen Auswertungen habe ich große Diskrepanzen in den versprochenen Vorteilen der verschiedenen Verbesserungen festgestellt.
Dies kann möglicherweise an der geringen Anzahl Iterationen der Baumsuche pro Zug gelegen haben.
Eine genauere Untersuchung der Verbesserungen in Abhängigkeit von der Anzahl der Iterationen wäre interessant.
Manche Verbesserungen entfalten womöglich erst aber einem gewissen Schwellenwert ihre volle Wirkung während andere nur bei sehr wenigen Iterationen wirklich effektiv sind.
Dafür wäre aber sehr viel mehr Zeit und eine effizientere Implementierung der Baumsuche in einer schnelleren Programmiersprache wie C oder Java nötig.

Ich habe mich bei meinen Betrachtungen der Transpositionen nur auf die Arbeit von Saffidine und Cazenave gestützt.
XXX haben in ihrem Paper ``Upper Confidence Bounds for Directed Acyclic Graphs'' einen parametrisierbaren Ansatz für Transpositonen vorgestellt.